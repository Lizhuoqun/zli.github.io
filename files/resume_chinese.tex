% resume.tex
%
% (c) 2002 Matthew Boedicker <mboedick@mboedick.org> (original author) http://mboedick.org
% (c) 2003 David J. Grant <dgrant@ieee.org> http://www.davidgrant.ca
% (c) 2007 Todd C. Miller <Todd.Miller@courtesan.com> http://www.courtesan.com/todd
% (c) 2009-2012 Derek R. Hildreth <derek@derekhildreth.com> http://www.derekhildreth.com 
%This work is licensed under the Creative Commons Attribution-NonCommercial-ShareAlike License. To view a copy of this license, visit http://creativecommons.org/licenses/by-nc-sa/1.0/ or send a letter to Creative Commons, 559 Nathan Abbott Way, Stanford, California 94305, USA.

% GENERAL NOTE:  There may be some notes specific to myself.  If you're only interested in my LaTeX source or it doesn't make sense, please disregard it.

\documentclass[letterpaper,11pt]{article}
\ifx\HCode\UnDef\else\hypersetup{tex4ht}\fi

%-----------------------------------------------------------
\usepackage{latexsym}
\usepackage[empty]{fullpage}
\usepackage[usenames,dvipsnames]{color}
\usepackage{verbatim}
\usepackage{hyperref}
\usepackage{ragged2e}
\usepackage{xeCJK}


\hypersetup{
    colorlinks,%
    citecolor=black,%
    filecolor=black,%
    linkcolor=black,%
    urlcolor=black 
    %urlcolor=mygreylink     % can put red here to better visualize the links
}
\urlstyle{same}
\definecolor{mygrey}{gray}{.85}
\definecolor{mygreylink}{gray}{.40}
\textheight=9.0in
\raggedbottom
\raggedright
\setlength{\tabcolsep}{0in}

\usepackage[a4paper,
            left=0.5in,right=0.5in,top=0.5in,bottom=0.5in,%
            footskip=.25in]{geometry}
\usepackage{enumitem}
\setlist[itemize]{leftmargin=0cm}
% \renewcommand{\labelitemii}{$\star$}

%-----------------------------------------------------------
%Custom commands
\newcommand{\resitem}[1]{\item  #1}
\newcommand{\resheading}[1]{{\large \colorbox{mygrey}{\begin{minipage}{\linewidth}{\textbf{#1 \vphantom{p\^{E}}}}\end{minipage}}}}
\newcommand{\ressubheading}[4]{
\begin{tabular*}{1.04\linewidth}{l@{\extracolsep{\fill}}r}
		\textbf{#1} & #2 \\
		\textit{#3} & #4 \\
\end{tabular*}\vspace{-6pt}}
\newcommand{\ressubsubheading}[2]{
\begin{tabular*}{1.04\linewidth}{l@{\extracolsep{\fill}}r}
	\textbf{#1} & #2 \\
\end{tabular*}\vspace{-6pt}}
%-----------------------------------------------------------

%-----------------------------------------------------------
%General Resume Tips
%   No periods!  Technically, nothing in this document is a full sentence.
%   Use parallelism by ending key words with the same thing,  i.e. "Coordinated; Designed; Communicated".
%   More tips on bottom of this LaTeX document.
%-----------------------------------------------------------


\begin{document}

\newcommand{\mywebheader}{
\begin{tabular*}{7in}{l@{\extracolsep{\fill}}r}
	\textbf{{\LARGE 尚晋}} & \href{mailto:jinshang@cs.cmu.edu}{jinshang@cs.cmu.edu}\\
	{\footnotesize \texttt{\href{https://jinshang.me}{jinshang.me} \hspace{1em} \href{https://github.com/js8544}{github.com/js8544} \hspace{1em} \href{https://linkedin.com/in/jinshang1997}{linkedin.com/in/jinshang1997}}}& +86 138 0455 3369 \\
	\end{tabular*}
\\
\vspace{0.1in}}

% CHANGE HEADER SOURCE HERE
\mywebheader

%%%%%%%%%%%%%%%%%%%%%%
\resheading{教育背景}
	\begin{description}
		\item
			\ressubheading{卡耐基梅隆大学 - 计算机科学院,CMU School of Computer Science}{匹兹堡, 美国}{计算机科学硕士, 4.05/4.00} {2019年8月-2020年12月}
			{ \footnotesize
				\begin{itemize}
					\resitem{\textbf{部分所选课程}: 高级深度学习(A+), 高级算法(A+),分布式系统, 数据库系统, 编译器设计,机器学习导论(博士), 理论计算机科学导论(博士),自动推理与可满足性检查(博士),密码学导论(博士)}
					\resitem{2020秋季学期分布式系统教学助理}
				\end{itemize}
				}
		\item
			\ressubheading{纽约大学阿布扎比分校,NYU Abu Dhabi}{阿布扎比,阿联酋}{数学与计算机科学学士,3.897/4.00} {2015年8月 - 2019年5月}
				{ \footnotesize
				\begin{itemize}
					% \resitem{\textbf{GPA}: 3.897/4.00; \textbf{Math Major GPA}: 4.00/4.00; \textbf{CS Major GPA}: 3.87/4.00; \textbf{Full scholarship} of \$350,000.}
                    \resitem{\textbf{部分所选课程}: 数据结构,算法,计算机系统结构,计算机网络,操作系统,软件工程,数学建模,数值方法,抽象代数I\&II,数学分析I\&II,微分方程}
                    \resitem{\textbf{研究方向}:超对称李代数与计算机代数系统,师从Dimitry Leites与Sofiane Bouarroudj}
				\end{itemize}
				}
\begin{comment}
\end{comment}
	\end{description} % End Education list

%%%%%%%%%%%%%%%%%%%%%%
\resheading{实习经历}
	\begin{description}
		\item 
			\ressubheading{谷歌}{湾区,美国}
				{实习软件开发工程师,谷歌云AI平台}{2020年5月-2020年8月}
				{ \footnotesize
				\begin{itemize}
					\resitem{参与了谷歌云AI平台的特征存储服务的设计与开发。特征存储是为机器学习团队实现规模化构造、分享与输出特征的云服务。}
					\resitem{设计并实现了特征存储的批量输出服务,旨在以高吞吐量一次性输出大批量历史特征数据,可直接用于训练机器学习模型。}
					\resitem{为特征存储批量输出编写了全面的单元测试与端到端测试,覆盖多种特征种类与实体类型的拼接并实现了用户请求验证。}
					\resitem{编写了特征存储的实例Jupyter Notebook,向谷歌内部与外部客户团队展示了特征存储在机器学习工作流的强大作用。}	
				\end{itemize}
				}
		\item 
			\ressubheading{腾讯}{深圳, 中国}
				{实习软件开发工程师,微信数据中心}{2019年6月-2019年8月}
				{ \footnotesize
				\begin{itemize}
					\resitem{使用FTRL在线学习算法实现了实时推荐系统,并使用微信PHXRPC框架进行模拟测试。达到5ms延迟与97.7\%准确率。}
					% \resitem{Programmed a sparse vector template class that supports operations in parallel computing using C++ STL and OpenMP.}
					\resitem{设计并开发了为腾讯Oceanus平台(基于Apache Flink)的特征工程工具,可以实现特征选取、拼接和重命名。该工具被使用到微信游戏推荐系统,其生成的基于用户数据、好友数据与好友关系数据的综合特征向量,可直接被用于训练推荐模型。}
					% \resitem{Replaced WeChat's batch-learning algorithm with the new feature toolkit and online recommendation algorithm; reduced total processing time by over 90\% for each post, resulting in 30\% more user clicks within the first month of deployment.}				
				\end{itemize}
				}
	\end{description}  % End Experience list
%%%%%%%%%%%%%%%%%%%%%%
% \let\oldhref\href
% \renewcommand{\href}[2]{\oldhref{#1}{\bfseries#2}}

\resheading{项目经历}
	\begin{description}
		\item
			\ressubsubheading{GoRaft: 用Go语言实现Raft一致性算法}{CMU $|$ 2019年10月-2019年11月}
				{ \footnotesize
				\begin{itemize}
					\resitem{在Go语言中使用RPC服务实现了Raft算法,一个通过领袖选举和日志复制实现一致性的分布式算法。}
				\end{itemize}
				}
		\item
			\ressubsubheading{分布式比特币挖矿系统}{CMU $|$ 2019年9月-2019年10月}
			{ \footnotesize
			\begin{itemize}
				\resitem{使用Go语言实现了存活序列协议(LSP),一个基于UDP的保证客户端-服务器可靠通讯的传输层协议。}
				\resitem{使用LSP实现了可规模化比特币挖矿系统,允许不限数量的挖矿机互相合作且保证负载均衡。}

			\end{itemize}
			}
		% \item
		% 	\ressubsubheading{BusTub Database Management System}{CMU $|$ Sep 2019 - Present}
		% 	{ \footnotesize
		% 	\begin{itemize}
		% 		\resitem{Implementing BusTub, a new disk-based relational DBMS.}
		% 	\end{itemize}
		% 	}
		% \item
		% 	\ressubsubheading{}{CMU $|$ Oct 2019 - Present}
		% 	{ \footnotesize
		% 	\begin{itemize}
		% 		\resitem{}
		% 		\resitem{}

		% 	\end{itemize}
		% 	}
		% \item
		% 	\ressubsubheading{Dynamic Memory Allocator (Malloc)}{CMU $|$ July 2019}
		% 		{ \footnotesize
		% 		\begin{itemize}
		% 			\resitem{Designed and developed memory allocator in C using segregated list and header compression averaging 75\% utilization.}
		% 		\end{itemize}
		% 		}
		% \item
		% 	\ressubsubheading{Course Equivalence Detector for NYU Global Campuses}{NYU $|$ Apr 2018 - May 2018}
		% 		{ \footnotesize
		% 		\begin{itemize}
		% 			\resitem{Developed a web scrapper to extract course information of 40,000 NYU courses across 14 global sites with Python Scrapy.}
		% 			\resitem{Generated word frequency vector for courses and detected similarity using TF-IDF and Naive Bayes with 85\% accuracy.}
		% 		\end{itemize}
		% 		}
		% \item
		% 	\ressubsubheading{PacMath: An Educational PacMan}{NYUAD $|$ Feb 2016 - May 2016}
		% 		{ \footnotesize
		% 		\begin{itemize}
		% 			\resitem{Developed a PacMan-like educational game that challenges junior students with their arithmetics and algebra skills.}
		% 		\end{itemize}
		% 		}
	\end{description}  % End Project list


%%%%%%%%%%%%%%%%%%%%%%
\resheading{研究经历}
	\begin{description}
		\item 
			\ressubsubheading{使用计算机代数系统研究复杂李超代数结构}{NYUAD $|$ Sep 2017 - May 2019}
				{ \footnotesize
				\begin{itemize}
					\resitem{设计了复杂度$O(n^2)$的算法,可用于在Mathematica中使用SuperLie库计算高维李代数的根和同调。}
					\resitem{使用新算法首次计算了多个李超代数的Duflo-Serganova函子和双重延拓。成果已发表为多篇论文。}
					% \resitem{Results are published in several papers; codes available at https://github.com/js8544/fast-cohomology-computation.}
					% \resitem{Coauthored 4 papers accepted to top mathematics journals including Experimental Math and Arnold Math Journal.}
				\end{itemize}
				}
		\item
			\ressubsubheading{无线物联网传感器的可感知性传输机制}{NYU $|$ June 2018 - Aug 2018}
				{ \footnotesize
				\begin{itemize}
					\resitem{使用动态规划方法为微型物联网传感器的设计了最优化传输算法,可以在减少50\%电量损耗的情况加增加3倍传输效用。}
					% \resitem{Tested proposed optimal algorithm which reduces power consumption by over 50\% while gaining 3 times more utility.}
					% \resitem{Coauthored a paper currently under review by IEEE Transactions on Cognitive Communications and Networking. }
				\end{itemize}
				}

	\end{description}  % End Experience list
%%%%%%%%%%%%%%%%%%%%%%
\resheading{部分论文}
\footnotesize(注:纯数学领域论文并无几作之分,作者按照姓氏首字母排序)
\begin{description}
	\item \footnotesize Sofiane Bouarroudj, Dimitry Leites, \textbf{Jin Shang}. \href{https://arxiv.org/abs/1904.09579}{"Computer-aided study of double extensions of restricted Lie superalgebras preserving the non-degenerate closed 2-forms in characteristic 2"}, \textit{Experimental Mathematics, 1-13, 1(2019).} 
	\item \footnotesize Sofiane Bouarroudj, Dimitry Leites, Alexander Lozhechnyk, \textbf{Jin Shang}. \href{https://arxiv.org/abs/1904.09578}{"The roots of exceptional modular Lie superalgebras with Cartan matrix"}, \textit{Arnold Mathematical Journal, 63-118, 6(2020).} 
	\item \footnotesize  \textbf{Jin Shang}, Muhammad Junaid Farooq, Quanyan Zhu. \href{https://arxiv.org/abs/1812.02615}{"Real-Time Transmission Mechanism Design for Wireless IoT Sensors with Energy Harvesting under Power Saving Mode"}, \href{https://arxiv.org/abs/1812.02615}{arxiv: 1812.02615}
\end{description}
%%%%%%%%%%%%%%%%%%%%%%
\resheading{技能}
\begin{description}
	\item \footnotesize \textbf{编程语言:} \textit{C/C++, Python, Java, GoLang,  SQL, Mathematica, Scala}
	\item \footnotesize \textbf{库与工具链:} \textit{C++ STL, Git, Numpy, Matplotlib, JUnit4, 谷歌内部工具链(Blaze, CitC, Guitar, \dots)}
\end{description}

% \newpage
%%%%%%%%%%%%%%%%%%%%%
% \begin{itemize}
% 	\setlength\itemsep{0.06cm}
% 	\item \footnotesize  Jin Shang, Muhammad Junaid Farooq, Quanyan Zhu \textbf{Real-Time Transmission Mechanism Design for Wireless IoT Sensors with Energy Harvesting under Power Saving Mode}. \textit{arxiv:1812.02615}, under review by \textit{IEEE Internet of Things Journal.}
% 	\item \footnotesize Sofiane Bouarroudj, Dimitry Leites, Jin Shang. \textbf{Computer-aided study of double extensions of restricted Lie superalgebras preserving the non-degenerate closed 2-forms in characteristic 2}. \textit{arxiv:1904.09579}, conditionally accepted to \textit{Experimental Mathematics.} 
% 	\item \footnotesize Sofiane Bouarroudj, Dimitry Leites, Alexander Lozhechnyk, Jin Shang. \textbf{The roots of exceptional modular Lie superalgebras with Cartan matrix}. \textit{arxiv:1904.09578}, conditionally accepted to \textit{Arnold Mathematical Journal.} 
	
% \end{itemize}

% %%%%%%%%%%%%%%%%%%%%%
% \resheading{AWARDS}
% \begin{itemize}
% 	\item \footnotesize \textbf{Silver Medal} \textit{Al-Kwarizmi International Mathematical Olympiad 2018}
% 	\item \footnotesize \textbf{Honorable Mention} \textit{North American Invitational Programming Contest 2018}
% \end{itemize}


\end{document}

